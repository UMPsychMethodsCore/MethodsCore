\chapter{SPM for EEG/MEG overview}
\label{ch:eeg_overview}
Unlike previous versions, SPM5 provides for
the analysis of EEG and MEG data. The initial main motivation for this big
leap came from the insight that any integration of modalities like
fMRI and EEG should be based on a common theoretical and practical
basis. Historically, research into fMRI and EEG/MEG models and
analysis has been quite divorced. The same is partially true for the
EEG and MEG field. It is our hope that SPM5 provides a common analysis
ground for modellers and experimentalists of both the PET/fMRI and 
EEG/MEG fields.
\\

SPM for EEG/MEG was primarily developed for the analysis of epoched
data. This is because we are mostly interested in experiments which
perturb the system with a designed stimulus. The analysis of
continuous data is typically performed for experiments without
designed stimuli like in sleep or epilepsy research. Note that although
SPM5 does not provide for an analysis of continuous data per se, many
SPM5 routines can be used for these data.
\\

SPM for EEG/MEG can be partitioned into four parts. These are (i)
preprocessing, (ii) projection to voxel-space/source reconstruction,
(iii) statistical analysis, and (iv) Dynamical Causal Modelling (DCM). 

The preprocessing functions provide for simple operations that are
standard in other software packages. 

The projection to voxel-space is a critical step. When using a source
reconstruction, it takes the analysis to brain space. But even when
using a simple projection to some 2D-sensor plane, we can then use SPM
functions and concepts that were developed for voxel-based data. The
projection to a 2D-plane is performed using a simple interpolation and
is mostly equivalent to widely used sensor-based analyses. The source
reconstruction to brain space is based on models that assume many
distributed dipoles in brain space. Solutions to these models
typically show dispersed activity and are well-suited for SPM
mass-univariate analysis approach.
\\

The statistical analysis is one of the strong points of SPM. PET/fMRI
users already familiar with the graphical representation of
general linear models won't have difficulties to use SPM for the
analysis of EEG/MEG data. SPM5 provides for a comprehensive range of
classical linear models that can be used to model the data. These
models are basically the same as used in the EEG/MEG field
for random effects analyses of multiple subjects. In SPM, we assume
that the data in voxel-space is a sampled version of some continuous Gaussian
random field. This allows us to use Random field theory (GFT)
for the correction of multiple comparisons to control family-wise
error over voxels. The GFT approach has the advantage that
super-threshold maxima are assessed for their significance, whereas
conventional approaches have to specify a-priori the locations of
expected activations.
\\

Dynamical Causal Modelling (DCM) is a departure from the mass-univariate
approach and is an extension of the SPM software package. DCM for ERP/ERFs is a
generative model for evoked responses. The observed data is modelled
as the spatiotemporal expression of a small hierarchical network that
responds to a stimulus. Differences between evoked responses due to different
stimuli are modelled as modulations of the coupling between specific
areas. Importantly, this approach is based on a neurobiologically
grounded model. This allows us to obtain parameter estimates that have
some physiological interpretation.
\\

The following chapters will go through all the EEG/MEG related
functionality of SPM5. All users will probably find the tutorial
useful for a quick start. A further detailed description of the
preprocessing functions is given in chapter
\ref{ch:eeg_preprocessing}. The 3D-source reconstruction is described
in chapter \ref{ch:eeg_imaging} and some dipole fitting technique in
chapter \ref{ch:eeg_ecd}. In chapter \ref{ch:eeg_stats}, we
guide you through the modelling of M/EEG data at the first and second
level of a hierarchical linear model. Finally, in chapter
\ref{ch:eeg_DCM}, we describe the graphical user interface for
dynamical causal modelling for evoked 
responses, i.e.~event-related potentials (ERPs) and event-related
fields (ERFs).

