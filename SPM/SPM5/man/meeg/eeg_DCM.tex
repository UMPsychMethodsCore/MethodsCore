\chapter{Dynamic Causal Modelling for evoked responses}
\label{ch:eeg_DCM}
Dynamic Causal Modelling for ERP/ERFs is described in
 \cite{david_dcm_erp} and \cite{sjk_dcm_erp}, see also\\
\href{http://www.fil.ion.ucl.ac.uk/spm/doc/biblio/}{\texttt{http://www.fil.ion.ucl.ac.uk/spm/doc/biblio/}}.
We recommend reading these two communications as a starter, because it will
enable you to better understand all the modelling options presented
in this chapter.

In summary, the goal of DCM is to explain evoked responses as the output of an
interacting network consisting of a few areas that receive an input
stimulus. The difference between two evoked responses that receive
comparable stimuli under two conditions is modelled as a 
modulation of some of the inter-areal connections
\cite{david_dcm_erp}. This interpretation 
of the ERP makes hypotheses about connectivity directly testable. For
example, one can ask, whether the difference between two
evoked responses can be explained by some top-down modulation of early areas
(Garrido et al., in preparation). Additionally, because DCM is framed
in a Bayesian way, one can also compute model evidences. These can be
used to compare alternative, equally plausible, models and decide
which model is the better one \cite{cdcm}.

DCM for ERP/ERFs takes the spatial forward model into account. To do
this, we parameterise the lead field, i.e.,~the spatial projection of source
activity to the sensors. In the present version, this is done by
assuming that each area is modelled by one equivalent current dipole
(ECD) \cite{sjk_dcm_erp}. In other words, DCM is used not only to
solve for the connectivity but, simultaneously, also for the spatial
parameters.

In the following, we will describe the graphical user interface
(GUI). The GUI allows to comfortably specify all parameters of a
model. If you want to specify lots of models, we recommend using a
batch script. An example of such a script
(\textit{DCM\_ERP\_example}), which can be adapted to your own
data, can be found in the \textit{man/example\_scripts/} folder of the
distribution.

\section{Calling DCM for ERP/ERF}
Currently, the GUI is hidden away as a toolbox. You can find it under
\textit{Toolboxes} $\to$ \textit{apierp}. The GUI is partitioned into 5
parts, going from the top to the bottom. The first part is about
loading and saving DCMs. The second part is about selecting some
data, the third is used to specify a spatial forward model, the
fourth is for specifying the connectivity model, and the last allows
you to estimate parameters and view results.

In general, you have to specify the data and model in a specific
order. The first stage is data selection, the second the spatial model,
followed by the connectivity model specification. This order is
necessary, because there are dependencies among the three parts that
would be hard to resolve without this order. However, at any time, you
can switch forth and back between model parts. Also, within each part,
you can specify information in any order you like.

\section{cd, load and save}
At the top of the GUI, you can \textit{cd} to a new working directory,
load existing DCMs or save a new one. In general, you can
\textit{save} and \textit{load} during model
specification at any time.

\section{Data selection}
In this part, you select the data. These can be either event-related
potentials or fields (i.e.~a data matrix (channels $\times$
peri-stimulus time), averaged over single trials). Currently, you can
only analyse one or two evoked responses in the same
model\footnote{Note however, that the theoretical framework allows
  analysis of any number of evoked responses}. To select
data, click on the button \textit{choose data}. Select an M/EEG
SPM-matfile which is the output of the SPM preprocessing. Alternatively, if your
data are in another format, you have to convert this file to the
SPM-format (see the preprocessing chapter of this manual to see how
this can be done). Below the \textit{choose data} button, you can
choose under \textit{nr} which of the evoked responses in the
SPM-matfile you want to model. For example, if you want to model the
second and third evoked response within a SPM-matfile, specify
indices 2 and 3. If your two evoked responses are in different files,
you have to merge these files first. You can do this with the SPM
preprocessing function \textit{merge} (\textit{spm\_eeg\_merge}),
s.~chapter \ref{ch:eeg_preprocessing}.
Under \textit{ms} you can specify the period of peri-stimulus time
which you want to model. After your data are loaded, the time-series of
all sensors are displayed in the SPM Graphics window. 

In DCM, we use a projection of the data to some subspace to reduce the
amount of data. Additionally, this data selection also serves as a
tool to model only the salient features of the
data. Currently, we are using a simple singular value decomposition
(SVD) to decompose the data. You can select the number of (first)
modes you want to keep. The default is 3, which we experimentally
found to be a good value for our own data.

Furthermore, you can choose whether you want to model the mean or
drifts of the data at sensor level. If you don't want any such terms,
select 0 for detrending. Otherwise, select the number of discrete cosine
transform terms you want to use to model the mean (1) or low-frequency
drifts ($> 1$).

If you are happy with your data, the projection and the detrending
terms, you can click on the $>$ (forward) button, which will bring you to
the next stage \textit{Spatial model specification}. Additionally,
when pressing the forward button, the reduced data is displayed in the
Graphics window. If you want to try other choices, you can press
the $<$ button (backward) button in the spatial modelling
part. This will take you back to the data selection part, where you
can change parameters and hit the forward button again.

\section{Spatial model specification}
With the present version of DCM, you have two options how to spatially
model your data. Either you compute the leadfield, for each area,
yourself, or you parameterise the leadfield using an equivalent
current dipole model.\\

For the first option, you need to choose
\textit{fixed} in the pull-down menu. Then click on \textit{load lead
field} and specify a matfile. This matfile must contain a matrix with
one column for each area. Each column must have one entry for each
channel. For an example, see \cite{david_dcm_erp}. Of course, you must have some
means of computing the leadfield for your experiment. Alternatively,
you can opt for the alternative and make the leadfield a function of
equivalent current dipoles (ECDs). The parameters of the ECDs
(location and orientation) can be estimated by DCM. To do this, you
need to either select \textit{ECD EEG} or \textit{ECD MEG}. For
\textit{ECD EEG}, you also need to select a sensor location file. This
can be either a Polehmus file or a matfile with a coordinate
matrix. Under \textit{names} specify the names of all areas, one name
per row. Under \textit{locations}, specify the locations of these
areas in MNI-space, again one location (three coordinates --- x y z) per
row. You can check the locations of dipoles by clicking on the
\textit{plot} button. This will visualize the dipole locations
overlaid on an MRI template in MNI space. Note that DCM uses by
default uninformative priors on 
dipole orientations, but rather tight priors on locations
\cite{sjk_dcm_erp}. One reason for the tight priors on locations is to
ensure that 
each dipole stays in its designated area and retains its meaning. The
prior location for each dipole can be found either by using available
anatomical knowledge or by relying on source reconstruction of
comparable studies. Also note that the prior location doesn't need to
be overly exact, because the spatial resolution of M/EEG is on a scale
of several millimeters. When you want to proceed to the next model
specification stage, hit the $>$ (forward) button and proceed
to the \textit{connectivity model specification}.


\section{Connectivity model specification}
Press \textit{specify connections} to get access to selecting your model's
connections between areas. There are 5 elements
which you need to go through. The first three are the intrinsic
connectivities. In DCM for ERP/ERF there are three types of connections,
\textit{forward}, \textit{backward} and \textit{lateral}. In each of
these matrices you specify a connection \textit{from} a source area
\textit{to} a target area. For example, switching on the element
$(2,\;1)$ in the intrinsic forward connectivity matrix means that you
specify a forward connection from area 1 to 2. Some people find the
meaning of each element slightly counter-intuitive, because the
column index corresponds to the source area, and the row index to the
target area\footnote{Currently, you can't model self-connections, which may
be introduced with a later SPM update.}.

In the present implementation, there is only one input allowed. This
input can go to any area, where it only goes to early
areas typically. You can select these receiving areas by selecting area indices in
the \textit{C} vector.

The \textit{B} matrix contains all gain modulations of intrinsic
connections. These modulations model the difference between the first
and second evoked response. In other words, the DCM explains two
evoked responses by explaining the first response by using the intrinsic
connections only. The 2nd response is modelled by modulating these
intrinsic connections by the weights in matrix \textit{B}. For
instance, if you want to allow modulations of forward connections
only, you switch on those connections in \textit{B} which are also
selected in the intrinsic forward connectivity matrix.

\section{Estimation}
When you are done with model specification, you can hit the
\textit{estimate} button in the lower left corner. DCM will first try
to save your DCM. Select a file to save to. After saving, DCM will
estimate model parameters. You can follow the estimation process by
observing the model fit in the output window. In the matlab command
window, you will see each iteration commented by iteration number,
free energy $F$, and the change of $F$ with respect to the
updated variance parameters.

Note that a DCM for evoked responses is more complex than a
DCM for fMRI. This means that more model parameters are used and,
consequently, the estimation process takes longer. Expect something
like 15 - 60 minutes, depending on model/data/computer specification.

\section{Results}After estimation finished, you can assess the results by choosing from
the pull-down menu at the bottom (middle). 

With \textit{ERPs channel} you can plot, for each mode, the data for
both evoked responses, and its fit by the model. 

When you select \textit{ERPs sources}, the dynamics of each area are
plotted. These corresponds to the (output) states of the dynamic
system \cite{david_dcm_erp}.

The option \textit{coupling (A)} will take you to a summary about the
posterior distributions of the intrinsic connectivities. In the upper
row, you see the posterior means for all intrinsic connectivities. As
above, element $(i,\; j)$ corresponds to a connection from area $j$
to $i$. In the lower row, you'll find, for each connection, the
probability that its posterior mean is different from the prior
mean, taking into account the posterior variance.

With the option \textit{coupling(B)} you can access the equivalent
posterior means for the gain modulations of the intrinsic
connectivities and their probability that they are unequal the prior
means.

With \textit{coupling(C)} you see the same summary of the posterior
distribution for the strength of the input into the input receiving
area. On the left hand side, DCM plots the posterior means for each
area. On the right hand side, you can see the corresponding
probabilities (s.~above).

The option \textit{Input} shows you the estimated input function. As
described by \cite{david_dcm_erp}, this is a gamma function with an
addition of some low-frequency terms.

With \textit{Response}, you can plot the selected data, i.e.~the modes
you have selected for DCM analysis.

And finally, with the option \textit{Dipoles}, DCM displays an
overlay of each dipole on an MRI template using the posterior means of
its 3 orientation and 3 location parameters.


 

