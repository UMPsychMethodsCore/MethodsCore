\chapter{EEG/MEG preprocessing --- Reference}
\label{ch:eeg_preprocessing}
In this chapter we will describe the purpose and syntax of all
SPM/MEEG preprocessing functions. These functions can be called either from
SPM's graphical user interface (GUI) or from the matlab command
line. For the command line, we follow the concept of providing only one input
argument to each function. This input argument is usually a structure (struct)
that contains all input arguments as fields. This has the advantage
that the input does not need to follow a specific input argument
order. If an obligatory input argument is missing, the function will
invoke the GUI and ask the user for the missing argument. When using
the GUI, a function is called without any input argument, i.e.~SPM
will ask for all input arguments. If using the command line, you
can specify all arguments in advance and use SPM/MEEG functions in
batch mode. We provided some sample batch script
(\textit{meeg\_preprocess}) in the \textit{man/example\_scripts/}
folder of the distribution.

\section{Conversion of data}
The conversion of data is necessary to read data from a native
machine-dependent format to a matlab-based, common SPM format. This
format stores the data in a *.dat file and all other information in a
*.mat file. The contents of the *.mat file is a single struct with
fields that contain all information about the data (described
further below). Currently, SPM can deal with a few formats (s.~below). 

\subsection{Converting data yourself}
If your format is not one of these, you need to convert the data
yourself. This might sound difficult, but it is actually easier than
most people think. If things go wrong, the SPM developer team is
usually quick to answer questions concerning help with the conversion
of data. What we can't do though is to provide a conversion routine
for every M/EEG format. The reason is that there are many formats
around, which also evolve over time. To support all these formats
would be simply too much work for us. 

To write a conversion routine, you (or a helpful colleague) need a
minimum knowledge of MatLab. To make things easier and to provide you
with a starting point, we wrote a generic conversion script that can
be easily modified to work with your specific data. You find this
script (\textit{meeg\_read\_data}) in the \textit{man/example\_scripts/}
folder of the distribution.

There are three parts to this script. In the first part you provide SPM
with information about your data, e.g.~sample rate, 
number of conditions, etc. The second part will read the actual data
into the MatLab workspace. To do this you'll need to write a a few 
lines of matlab. This can be easy, if your data is in ASCII
format. It's more difficult, when the data is still in its native (binary)
format. In that case you must know the file specification. Alternatively, you 
might be fortunate and find free third-party MatLab-software, somewhere on
the internet, that does this job for you! In the third part, the data
and all information is converted to the SPM format.

The final step is to generate a channel template
file. This is necessary to determine the coordinates of the channels
in 2D-space. These coordinates are needed for viewing data and
projection to voxel-space\footnote{If you don't want to look at your
  data, or project to 2D voxel-space, you can actually proceed without
  this channel template file.}. See below how such a channel template file
can be generated. If you're using a standard setup, it's likely that
we or someone else have already provided such a file.

\section{Converting CNT or BDF files}
The conversion routine can be started either by using the GUI
(Convert) or by calling the function \textit{spm\_eeg\_convert2mat}.
This function is simply a wrapper function that calls the appropriate
conversion function.

\subsection{Syntax}
\textit{D = spm\_eeg\_converteeg2mat(S)}
\\

\subsubsection{Input}
The input struct {\it S} is optional and has the following optional fields:

\begin{tabular}{lcp{9cm}}
{\bf fmt} & - & string that determines type of input file. Currently,
this string can be either 'CNT', 'BDF', 'EGI-txt'\\
{\bf Mname} & - & char matrix of input file name(s)\\
{\bf Fchannels} & - & String containing name of channel template file 
\end{tabular}

\subsubsection{Output}
The output struct {\it D} contains the header struct of the converted
file. This struct has been written to a *.mat file of the same name as
the converted file. The data has been written to a corresponding *.dat
file.

\section{BDF data}
The Biosemi Data format (BDF) can be converted with the function {\it
spm\_eeg\_rdata\_bdf}. There is no explicit reference electrode,
because the Biosemi system uses reference-free measurements. Nearly
all information is contained in the raw *.bdf file. The only
information that is not in the file is the actual usage of the 8
external channels. Typically these are used for EOG and some reference 
measurement. These information must be supplied to SPM. Keep in mind
that a false declaration of external channels can severely degrade
the quality of your data.

The conversion routine can be started by calling the function
\textit{spm\_eeg\_rdata\_bdf}. 

\subsection{Syntax}
\textit{D = spm\_eeg\_rdata\_bdf(S)}
\\

\subsubsection{Input}
The input struct {\it S} is optional and has the following optional fields:

\begin{tabular}{lcp{9cm}}
{\bf Fdata} & - & filename of bdf-file\\
{\bf Fchannels} & - & String containing name of channel template
file\\
{\bf Cheog} & - & One or two indices of external channels used for
HEOG. Valid indices lie between 1 and 8.\\
{\bf Cveog} & - & indices (1 - 8) of external channels used for
VEOG. Valid indices lie between 1 and 8.\\
{\bf Creference} & - & indices (1 - 8) of external channels used for
reference. Valid indices lie between 1 and 8.
\end{tabular}

\subsubsection{Output}
The output struct {\it D} contains the header struct of the converted
file. This struct has been written to a *.mat file of the same name as
the converted file. The data has been written to a corresponding *.dat
file.

\section{CNT data}
The neuroscan CNT format can be converted with the function {\it
spm\_eeg\_rdata}. Nearly
all information is contained in the raw *.cnt file. The only
information that is not in there is about the reference used. SPM will
ask you explicitly about the reference. However, if you
don't want to re-reference your data at a later stage, you don't need
to supply information about the reference electrode. The same is true,
if you want to re-reference, but don't want to transform the reference
channel again to an EEG channel. For example, this is the case when
the reference were the earlobes.

The conversion routine can be started by calling the function
\textit{spm\_eeg\_rdata}.

\subsection{Syntax}
\textit{D = spm\_eeg\_rdata(S)}
\\

\subsubsection{Input}
The input struct {\it S} is optional and has the following optional fields:

\begin{tabular}{lcp{9cm}}
{\bf Fdata} & - & filename of CNT-file\\
{\bf Fchannels} & - & String containing name of channel template file\\
{\bf reference} & - & name of reference channel. If you want to make this
channel an EEG channel at a later re-referencing, you need to supply
the exact name of the channel. If you just want to store the reference
name (e.g.~earlobes), just enter any descriptive text.
\end{tabular}

\subsubsection{Output}
The output struct {\it D} contains the header struct of the converted
file. This struct has been written to a *.mat file of the same name as
the converted file. The data has been written to a corresponding *.dat
file.

\section{The MEEG SPM format}
The SPM-format stores the binary data in a *.dat file. All header
information are stored in a *.mat file. This *.mat file contains a
single struct named {\it D} which contains several fields. Note that
the data should always be read using the routine {\it spm\_eeg\_ldata},
see section \ref{sec:ldata}. In the following, we will first describe
all single-element entries, and then all entries that are itself structs.

\begin{tabular}{lcp{9cm}}
{\bf Nchannels} & - & The number of channels. This number also
includes channels like EOG or other external channels\\
{\bf Nevents} & - & The number of epochs\\
{\bf Nsamples} & - & The number of time bins in one epoch\\
{\bf Radc} & - &  The sampling rate measured in Hertz\\
{\bf fnamedat} & - &  The name of the *.dat file without leading
path\\
{\bf fname} & - &  The name of the *.mat file without leading path\\
{\bf path} & - &  The path to the directory where the *.mat and *.dat
file are stored\\
{\bf datatype} & - &  The datatype with which the data in the *.dat
file is stored. Possible datatypes are 'int16' and 'float'\\
{\bf data} & - & This is a spm\_file\_array struct that contains the
memory mapped data. For epoch data this is effectively a
three-dimensional array of the dimensions {\it Nchannels} $\times$
{\it Nsamples} $\times$ {\it Nevents}.\\
{\bf scale} & - &  A matrix with internally used scaling values for
the memory mapping of data. For documentation, see the directory
\textit{file\_array} in the SPM main directory.\\
{\bf modality} & - &  A string that is (currently) either 'EEG' or
'MEG' and describes the type of data in the file \\
{\bf units} & - &  A string that determines the units of the data in
tex-format, e.g.~$\mu V$ for micro V \\
\end{tabular}

\subsection{channels}
The substruct {\it channels} contains all channel-related
information.

\begin{tabular}{lcp{9cm}}
{\bf ctf} & - & The name of a channel template file (CTF) without leading
path. It is assumed that the CTF is located in the EEGtemplates
sub-directory of the SPM5 main directory. This file
contains standard channel names for a given setup and their
coordinates on a 2D plane. When converting a file to the SPM-format, a
link is made to a CTF. Identification of channels in the data file is
via the channel names. (See also sec.~\ref{sec:ctf})\\
{\bf Bad} & - & An index vector of bad channels\\
{\bf name} & - & A cell vector of channel names\\
{\bf eeg} & - & The indices of actual EEG channels. For example, these
exclude the EOG channels.\\ 
{\bf order} & - & An index vector that maps from the data order of
channels to the corresponding channel in the CTF.\\
{\bf heog} & - & The channel index of the HEOG channel.\\
{\bf veog} & - & The channel index of the VEOG channel.\\
{\bf reference} & - & If available, this is an index of the reference
channel in the order of the CTF. Otherwise this is 0.\\
{\bf ref\_name} & - & If available, the name of the reference
channel. This actually does not need to be a valid channel name, but
is just used as a reminder for the user (e.g. 'earlobes').\\
{\bf thresholded} & - & A cell vector that contain channel indices (in
data order) of epochs with data surpassing some threshold. This is
usually generated by the {\it spm\_eeg\_artefact} function.\\
\end{tabular}

\subsection{events}
The substruct {\it events} contains information related to the epochs.

\begin{tabular}{lcp{9cm}}
{\bf code} & - & A vector which contains event numbers for each event.
These were read during the conversion from the event channel of the
raw data.\\
{\bf time} & - & A vector which contains the timing of stimulus
presentation for each event (measured in time bins). This is used for
epoching the data.\\
{\bf start} & - & The number of time bins before onset of the
stimulus\\
{\bf stop} & - & The number of time bins after onset of the
stimulus\\
{\bf Ntypes} & - & The number of different event types\\
{\bf reject} & - & A vector which for each event a 0 or 1's indicating
whether this trial was rejected or not.\\
{\bf repl} & - & A vector with the number of single trials which were
used for each event-type by the averaging function.\\
\end{tabular}


\subsection{filter}
The substruct {\it filter} contains information about filtering the
data. This struct is usually filled by the function {\it
  spm\_eeg\_filter}.

\begin{tabular}{lcp{9cm}}
{\bf type} & - & The name of the used filter\\
{\bf band} & - & 'lowpass' or 'bandpass'\\
{\bf PHz} & - & The cutoff of the filter in Hertz\\
{\bf para} & - & A cell vector with filter parameters. See the matlab
function {\it filter} for a description of what these parameters are.
\end{tabular}

\subsection{threshold}
The substruct {\it threshold} contains information about thresholding
  the data. This struct is usually filled by the function {\it
  spm\_eeg\_artefact}.

\begin{tabular}{lcp{9cm}}
{\bf External\_list} & - & Indicator (0/1) whether external information
was used which trials were artefactual or clean.\\
{\bf threshold} & - & The threshold used in microVolt
\end{tabular}


\section{Reading of data}
\label{sec:ldata}
Once the data is in SPM-format, it can be read into matlab. This should
be done using the routine {\it spm\_eeg\_ldata}. (Note: If you only
work with the GUI, you won't need to call this function.) The
routine will mainly do two things. First, it will load the header struct in the
*.mat-file. Secondly, it will memory map the data in the *.dat file to
a field in this struct. The memory mapped data can be addressed like a
matrix which is convenient for accessing the data in a random access
way. However, a word of caution: If you write new values to the
D.data-matrix, the matrix is not only changed in the matlab variable
(in memory), but also physically on the hard disk. 

This function can only be called via matlab command line.

\subsection{Syntax}
\textit{D = spm\_eeg\_ldata(P)}
\\

\subsubsection{Input}
The input string {\it P} is optional and contains the file name of the
*.mat file.

\subsubsection{Output}
The output struct {\it D} contains all header information about the
data. The data are memory mapped and can be accessed as the field {\it data}.

\section{The channel template file}
\label{sec:ctf}
The channel template file is SPM's way of connecting acquired channel
data to a spatial location. The locations of channels are typically not
contained in the MEEG raw data file. A channel template
file contains channel names for a given setup and their locations in
some coordinate system. All channel template files are contained in
the subdirectory {\it EEGtemplates} in the SPM5-directory.
During the initial conversion of data each channel is
identified by its name and mapped to the corresponding channel
location contained in the channel template file. If a channel's name
is not contained in the user-specified channel template file, a
warning is issued. Many warnings usually mean that the wrong channel
template file for a specific setup was selected. Note that
even if the mapping from channels to their locations were not
identified correctly, it is still possible to perform preprocessing
operations (e.g.~epoching, filtering, etc.) on the converted
data. However, the channels' locations are needed for display and
mapping to voxel space. 

Currently, the channel template files in SPM5b all map into some standard
2D space on the scalp. This is useful for mapping multiple subjects'
data to a standard space and performing SPM analyses in 2D scalp
space. Future updates of SPM5b will supply channel
template files that map to a 3D sensor space, which is critical for
3D source reconstruction. This 3D space can be some standard space
which might be useful for MEG data and EEG data acquired with a
cap. Alternatively, one can also use digitized sensor positions as
locations, e.g. acquired with a Polyhmus system.

\subsection{Structure}
A channel template file (CTF) is a mat-file that contains four
variables:\\

\begin{tabular}{llcp{9cm}}
{\bf Nchannels} & &  - & The number of channels known to the CTF\\
{\bf Cnames}&  & - & A cell vector of channel names. Each cell can
contain either a string or a cell vector of strings. The latter allows
to have multiple versions of a given channel name. Case can be
ignored, i.e.~it doesn't matter whether channel names are in small or
captial letters.\\
{\bf Cpos} & & - & A $2 \times Nchannels$-matrix of channel
coordinates on a 2D plane. In $x$- and $y$-direction the minimum
coordinate must be $\leq 0.05$ and the maximum coordinate
must be $\geq 0.95$. \\ 
{\bf Rxy} & & - & A factor that determines the width of the display
plots to their heigth when displaying the data. Standard is 1.5. \\
\end{tabular}

\subsection{Creating your own channel template file}
The channel template file is important for using SPM's full
functionality for MEEG data. The channel template files contained in
the EEGtemplates directory are the ones that we or our collaborators
found useful. Other groups will need different channel template files,
because they might have different setups, i.e.~they use different
channel names and/or different channel coordinates. Note that if a
specific setup is just a subset of channels of an existing setup, the
CTF of the full setup can be used.  

If a new channel template file is needed, this can be simply created by
saving the variables {\it Nchannels, Cnames, Cpos and Rxy} to a new
channel template file. Typically this would be done by running a
script that creates these four variables and saves them to a file. The
creative bit is to list the actual coordinates of the channels on a
2D plane. We found two feasible ways for doing this. The first is to
note that many electrode setups consist of electrodes sitting on
concentric rings equidistant to other electrodes on each ring. Such a 
setup can be programmed as a script which places electrodes on each of
these rings. A second way is that at least some producers of EEG caps
provide coordinates for specific setups in 3D space. For example,
have a look at
http://www.easycap.de/easycap/e/downloads/electrode\_sites\_coordinates.htm.
The projection to 2D coordinates could be done by first using Matlab's
{\it sph2cart} function to transform to Cartesian coordinates. This is
followed by applying the subfunction {\it CartToFlat} of {\it
  spm\_eeg\_DrawSV} (SPM5 DipoleFit toolbox) to the Cartesian coordinates.
We provided an example script (\textit{make\_Easycap\_montage1}) in the
\textit{man/example\_scripts/} folder of the distribution to illustrate
this process.

\section{Epoching the data}
Epoching cuts out little chunks of data and saves them as 'single
trials'. For each stimulus onset, the epoched trial starts at some
user-specified pre-stimulus time and end at some post-stimulus time,
e.g.~from -100 to 400 milliseconds in peri-stimulus
time. The epoched data is also baseline-corrected, i.e.~the mean of
the pre-stimulus time is subtracted from the whole trial. The
resulting event codes are the same as saved in the *.mat file. One can
re-code events by supplying a vector of event codes.

The epoching routine can be started either by using the GUI
(Epoching) or by calling the function \textit{spm\_eeg\_epochs}.

\subsection{Syntax}
\textit{D = spm\_eeg\_epochs(S)}
\\

\subsubsection{Input}
The input struct {\it S} is optional and has the following optional fields:

\begin{tabular}{llcp{9cm}}
{\bf D} & &  - & filename of MEEG mat file\\
{\bf events}&  & - & a struct containing the following fields\\
& {\bf start} & - & pre-stimulus start of epoch[ms]\\
& {\bf stop} & - & post-stimulus end of epoch[ms]\\
& {\bf types} & - & vector of event types to extract\\
& {\bf Inewlist} & - & indicate (0/1) to use new list of event codes\\
& {\bf Ec} & - & vector of new event codes
\end{tabular}

\subsubsection{Output}
The output struct {\it D} contains the header struct of the epoched
file. This struct has been written to a *.mat file of the same name as
the input file, but prepended with {\it e\_}. The data has been
written to a corresponding *.dat file.


\section{Filtering the data}
Continuous or epoched data can be filtered with a low- or
bandpass-filter. SPM uses a Butterworth filter to do this. Phase
delays are minimised by using matlab's {\it filtfilt} function which
filters the data twice, forwards and
backwards. SPM's filter function {\it spm\_eeg\_filter} uses matlab's
signal processing toolbox. If you don't have this toolbox, you cannot
filter your data using SPM. 

The filter routine can be started either by using the GUI
(Filter) or by calling the function \textit{spm\_eeg\_filter}.

\subsection{Syntax}
\textit{D = spm\_eeg\_filter(S)}
\\

\subsubsection{Input}
The input struct {\it S} is optional and has the following optional fields:

\begin{tabular}{llcp{9cm}}
{\bf D} & & - & filename of MEEG mat file\\
{\bf filter}&  & - & a struct containing the following fields\\
& {\bf type} & - & type of filter, currently must be 'butterworth'\\
& {\bf band} & - & a string, 'lowpass' or 'bandpass' \\
& {\bf PHz} & - & one (lowpass) or two (bandpass) cut-offs [Hz]\\
\end{tabular}

\subsubsection{Output}
The output struct {\it D} contains the header struct of the filtered
file. This struct has been written to a *.mat file of the same name as
the input file, but prepended with {\it f}. The data has been
written to a corresponding *.dat file.


\section{Artefact detection and rejection}
Some trials are likely to not only contain neuronal signals of
interest, but also signal from other sources like eye movements or
muscular activity. These signal components are referred to as
artefacts. In SPM, we use only two simple automatic artefact detection
schemes. The first is thresholding the data and the second is robust
averaging. One can also choose to detect artefacts manually by
visualizing each trial (see below). Another option is to use a
more shophisticated artefact detection approach (implemented by some
other software) and supply that information to SPM. 

Thresholding the data is done in two passes. In the first pass, SPM
detects all instances for which the treshold was passed by the
absolute value for a channel and single trial. If a channel has more
than a certain percentage of artefactual trials, it is defined as a
bad channel. In a second pass the thresholding is repeated, but
without taking into account bad channels. A trial for which the
absolute data surpasses the treshold in some channel (excluding bad
channels) is considered artefactual.

The function only indicates which trials are artefactual or clean and
subsequent processing steps (e.g.~averaging) will take this
information into account. However, no data is actually removed from
the *.dat file.

The artefact routine can be started either by using the GUI
(Artefacts) or by calling the function \textit{spm\_eeg\_artefact}.

\subsection{Syntax}
\textit{D = spm\_eeg\_artefact(S)}
\\

\subsubsection{Input}
The input struct {\it S} is optional and has the following optional fields:

\begin{tabular}{llcp{9cm}}
{\bf D} & & - & filename of MEEG mat file\\
{\bf thresholds}&  & - & a struct containing the following fields\\
& {\bf External\_list} & - & indicate (0/1) to use external arefact list\\
& {\bf out\_list} & - & index vector of artefactual trials \\
& {\bf in\_list} & - & index vector of clean trials\\
& {\bf Check\_Threshold} & - & indicate (0/1) whether to threshold
channels\\
& {\bf threshold} & - & threshold to use: can be either a scalar which
is the treshold for all channels, or a vector of channel-wise
thresholds\\ 
& {\bf in\_list} & - & index vector of clean trials\\
{\bf artefact}&  & - & a struct containing the following fields\\
& {\bf weighted} & - & indicate (0/1) whether to use robust averaging\\
& {\bf wtrials} & - &  indicate (0/1) whether to use robust averaging
across trials\\
\end{tabular}

\subsubsection{Output}
The output struct {\it D} contains the header struct of the artefact-detected
file. This struct has been written to a *.mat file of the same name as
the input file, but prepended with {\it a}. The data has been
written to a corresponding *.dat file.

\section{Downsampling}
The data can be downsampled to any sample rate. This is useful if the
data was acquired at a higher sampling rate than one needs for
making inferences about low-frequency components. SPM's downsampling
routine uses the matlab function {\it resample}, which is part of
matlab's signal processing toolbox. If you don't have this toolbox, you cannot
downsample your data using SPM. 

The downsampling routine can be started either by using the GUI
(Other/downsample) or by calling the function
\textit{spm\_eeg\_downsample}.

\subsection{Syntax}
\textit{D = spm\_eeg\_downsample(S)}
\\

\subsubsection{Input}
The input struct {\it S} is optional and has the following optional fields:

\begin{tabular}{llcp{9cm}}
{\bf D} & & - & filename of MEEG mat file\\
{\bf Radc\_new} &  & - & the new sampling rate in Hertz
\end{tabular}

\subsubsection{Output}
The output struct {\it D} contains the header struct of the downsampled
file. This struct has been written to a *.mat file of the same name as
the input file, but prepended with {\it d}. The data has been
written to a corresponding *.dat file.

\section{Rereferencing}
When you acquired data to a certain reference, you can simply
re-reference the data to another channel or to the average over a set
of channels. Bad channels are excluded from an average reference. If
there was only a single reference channel before, one can add it again
to the data. The rereferencing routine displays the indices of all
channels of the data as a help to decide which indices to select as a
new reference.

The rereferencing routine can be started either by using the GUI
(Other/rereference) or by calling the function
\textit{spm\_eeg\_rereference}.

\subsection{Syntax}
\textit{D = spm\_eeg\_rereference(S)}
\\

\subsubsection{Input}
The input struct {\it S} is optional and has the following optional fields:

\begin{tabular}{llcp{9cm}}
{\bf D} & & - & filename of MEEG mat file\\
{\bf newref}&  & - & a struct containing the following fields\\
\end{tabular}

\subsubsection{Output}
The output struct {\it D} contains the header struct of the rereferenced
file. This struct has been written to a *.mat file of the same name as
the input file, but prepended with {\it R}. The data has been
written to a corresponding *.dat file.


\section{Grand mean}
The grand mean is usually understood as the average of ERPs over
subjects. The grand mean function in SPM is typically used to do
exactly this, but can also be used to average over multiple
EEG files, e.g.~multiple sessions of a single subject. The averaged
file will be written into the same directory as the first selected
file.

The grand mean routine can be started either by using the GUI
(Other/grand mean) or by calling the function
\textit{spm\_eeg\_grandmean}.

\subsection{Syntax}
\textit{D = spm\_eeg\_grandmean(S)}
\\

\subsubsection{Input}
The input struct {\it S} is optional and has the following optional fields:

\begin{tabular}{llcp{9cm}}
{\bf P} & & - & filenames of M/EEG mat files (char matrix)\\
\end{tabular}

\subsubsection{Output}
The output struct {\it D} contains the header struct of the averaged
file. This struct has been written to a *.mat file of the same name as
the input file, but prepended with {\it g}. The data has been
written to a corresponding *.dat file.


\section{Merge}
Merging several MEEG files can be useful for concatenating multiple
sessions of a single subject. Another use is to merge files and then
use the display tool on the concatenated file. This is the preferred
way in SPM to display data together that is split up into several
files. The merged file will be written into the same directory as the
first selected file.

The merge routine can be started either by using the GUI
(Other/merge) or by calling the function
\textit{spm\_eeg\_merge}.

\subsection{Syntax}
\textit{D = spm\_eeg\_merge(S)}
\\

\subsubsection{Input}
The input struct {\it S} is optional and has the following optional fields:

\begin{tabular}{llcp{9cm}}
{\bf P} & & - & filenames of MEEG mat files (char matrix)\\
\end{tabular}

\subsubsection{Output}
The output struct {\it D} contains the header struct of the merged
file. This struct has been written to a *.mat file of the same name as
the input file, but prepended with {\it c}. The data has been
written to a corresponding *.dat file.


\section{Time-frequency decomposition}
\label{sec:tf}
The time-frequency decomposition is performed by using a continuous
Morlet wavelet transform. The result is written as two result files,
one contains the instantaneous power and the other the phase
estimates. One can select the channels and frequencies for which power
and phase should be estimated. Optionally, one can apply a baseline
correction to the power estimates, i.e.~the mean power of the
pre-stimulus time is subtracted from the power estimates.

The time-frequency decomposition routine can be started either by
using the GUI (Other/time-frequency) or by calling the function
\textit{spm\_eeg\_tf}.

\subsection{Syntax}
\textit{D = spm\_eeg\_tf(S)}
\\

\subsubsection{Input}
The input struct {\it S} is optional and has the following optional fields:

\begin{tabular}{llcp{9cm}}
{\bf D} & & - & filename of MEEG mat file\\
{\bf frequencies} & & - & vector of frequencies [Hertz] at which
decomposition is performed\\
{\bf rm\_baseline} & & - & indicate (0/1) whether baseline should be
substracted\\
{\bf Sbaseline} & & - & start and stop of baseline (in time bins)\\
{\bf channels} & & - & indices of channels for which to perform
time-frequency decomposition\\
{\bf Mfactor} & & - & the so called Morlet wavelet factor, defaults to
7.\\
\end{tabular}

\subsubsection{Output}
The output struct {\it D} contains the header struct of the phase
information. This struct has been written to a *.mat file of the same
name as the input file, but prepended with {\it t2\_}. The data has
been written to a corresponding *.dat file. The power data has been
written to a file prepended with {\it t1\_}.

\section{Averaging}
Averaging of the single trial data is the crucial step to obtain the
ERP. By default, when averaging single trial data, single trials are
averaged within trial type. Power data of single trials (see
sec.~\ref{sec:tf}) can also be averaged.

The averaging routine can be started either by using the GUI (average)
or by calling the function \textit{spm\_eeg\_average}.

\subsection{Syntax}
\textit{D = spm\_eeg\_average(S)}
\\

\subsubsection{Input}
The input struct {\it S} is optional and has the following optional fields:

\begin{tabular}{llcp{9cm}}
{\bf D} & & - & filename of MEEG mat file\\
\end{tabular}

\section{Linear combinations of epochs}
As an extension to the averaging functionality, SPM can also be used
to compute linear combinations of single trials 
or epochs. For example, you might be interested in computing the
difference between two ERPs. This can be done by calling the function
\textit{spm\_eeg\_weight\_epochs}. 

\subsection{Syntax}
\textit{D = spm\_eeg\_weight\_epochs(S)}
\\

\subsubsection{Input}
The input struct {\it S} is optional and has the following optional fields:

\begin{tabular}{llcp{9cm}}
{\bf D} & & - & filename of MEEG mat file\\
{\bf c} & & - & a weight (contrast) matrix with dimensions
$N_{contrasts} \times N_{epochs}$. Each row of $c$ contrains one contrast
vector. For a simple difference between two ERPs use $[-1 \; 1]$.
\end{tabular}

\subsubsection{Output}
The output struct {\it D} contains the header struct of the averaged
file. This struct has been written to a *.mat file of the same name as
the input file, but prepended with {\it m}. The data has been
written to a corresponding *.dat file.

\section{Mixing of channels}
SPM can also be used to compute the mixing of channels by a square
matrix. For example, we found this useful for computing a weighting
of the data with an independent component analysis (ICA) mixing
matrix. You can do this by calling the function
\textit{spm\_eeg\_weight\_channels}.

\subsection{Syntax}
\textit{D = spm\_eeg\_weight\_channels(S)}
\\

\subsubsection{Input}
The input struct {\it S} is optional and has the following optional fields:

\begin{tabular}{llcp{9cm}}
{\bf D} & & - & filename of MEEG mat file\\
{\bf W} & & - & a mixing matrix with dimensions $N_{channels} \times
N_{channels}$. Hint: If you call the function without arguments, prepare a
variable that contains this matrix.
\end{tabular}

\subsubsection{Output}
The output struct {\it D} contains the header struct of the averaged
file. This struct has been written to a *.mat file of the same name as
the input file, but prepended with {\it w}. The data has been
written to a corresponding *.dat file.

\section{Weighting of the time-series}
You can use SPM to multiply your
data, over peri-stimulus time, by some weighting function. For
example, we found this useful for removing stimulus-related artefacts
due to an electrical impulse at peri-stimulus time 0. The weighting
would be a function over peri-stimulus time consisting of 1s
everywhere, except for time 0, where you would remove data by putting
in a 0. You can do this by calling the function
\textit{spm\_eeg\_weight\_time}. 

\subsection{Syntax}
\textit{D = spm\_eeg\_weight\_time(S)}
\\

\subsubsection{Input}
The input struct {\it S} is optional and has the following optional fields:

\begin{tabular}{llcp{9cm}}
{\bf D} & & - & filename of MEEG mat file\\
{\bf weight} & & - & a weighting function (vector) with length peri-stimulus
time points. Hint: If you call the function without arguments, prepare a
variable that contains this vector.
\end{tabular}

\subsubsection{Output}
The output struct {\it D} contains the header struct of the averaged
file. This struct has been written to a *.mat file of the same name as
the input file, but prepended with {\it w}. The data has been
written to a corresponding *.dat file.

\section{Displaying data}
SPM can be used to display epoched data. The viewer is called by
choosing the EEG/MEG modality and clicking on the {\it M/EEG} entry in
the {\it Display} menu. After selecting an epoched M/EEG file in SPM
format, the viewer displays all channels of the first trial or trial
type. The position of the channels is taken from the channel template
file (see above). 

Navigation through trials or trial types is either by the trial
slider or by the trial listbox. The scaling of the displayed data (for
EEG: $\mu$Volt, for MEG: in $10^2$ femto Tesla) can be changed by using
the scaling slider. Up to four trials or trial types can be plotted at
the same time by using the shift or ctrl-button while selecting files
with the left mouse button in the listbox.

Single trials can be classified as either artefactual or clean by
pressing the {\it Reject} button. This information can be saved to the
*.mat file by pressing the {\it Save} button.

A left-clicking on a single channel plot will plot the time-series of
this channel in much more detail in a pop-up figure. Another
left-click on the (small) channel plot will close the pop-up figure
again. 

The topography at a specific time point can be displayed either in 2D
or in 3D by clicking on the {\it Topography} button and selecting a
peri-stimulus time and choosing between 2D/3D. In this display, bad
channels will not be interpolated, but no data is plotted in the
location of the bad channel.

The set of displayed channels can be changed by clicking on the {\it
channel} button. This is useful for (i) faster plotting of single
trial data by choosing less channels to display and (ii) having larger
plots in the display tool. In the channel select tool, you can click on
the channel to select or deselect this channel. Alternatively, you can
also use the listbox with shift/ctrl to select or deselect
channels. Channel selection can be saved and loaded to
mat-files. Pressing {\it Ok} will confirm your selected channels and
update your display.

