\chapter{Spatial projections}
\label{ch:eeg_source}
After pre-processing the M/EEG data, the data is projected to voxel
space. This is a critical departure from most standard analyses for
ERP data which typically analyse data in channel space. The main
motviation to project data to voxel space is our overall goal to integrate
M/EEG with other modalities like functional magnetic resonance imaging
(fMRI). A minimum requirement for integration is that both kinds of
data are in the same space. Naturally, this space is the 3D-brain
space, because it is here where neuronal sources are active and cause
observed M/EEG activity in the sensors. 

Projection to brain space, i.e.~source reconstruction, is usually done
using one of two approaches. The first is based on the assumption that
the potential fields measured on the scalp can be explained by a few
equivalent current dipoles (ECD). Although this is a very parsimonious
model and widely used in the M/EEG community, this is not the approach
we are going to use for mass-univariate analysis in SPM. Rather we use
a second approach, which models the observed sensor data by hundreds to
thousands of dipoles in brain space. Such a model has many more
parameters than the sparse dipole model, but parameters can be
estimated efficiently using Baysian techniques with informed
functional and anatomical priors. The solutions from these distributed
models lead to reconstructed sources that can be extended and don't
need to be focal like in the ECD approach.

However, note that we do not exclude ECD models from the SPM
software. Although we won't use ECD models for the mass-univariate
analysis, we are actively pursuing dynamical causal modelling based on
spatial forward models that use ECD representations (see chapter
??). Furthermore, there is a toolbox available for ECD reconstruction
using classical techniques (see below). One can use this toolbox to derive
dipolar solutions for multiple subjects and analyse the reconstructed
source activity outside of SPM (i.e.~the mass-univariate approach).

In cases when one is not interested in source reconstruction, one can
instead project to the 2D scalp surface. This procedure basically
leads to a conventional sensor-based analysis. 


\section{Distributed linear models}
Jermemie's bit.

\section{Equivalent current dipoles}
Christophe's bit.

\section{Interpolation on scalp}
The interpolation of sensor data in 2D voxel space is mostly equivalent to
a sensor-based analysis. The critical difference to conventional
analysis approaches lies hidden in the treatment of the factor {\it
space}. In the ERP community, the factor space is traditionally 
considered a factor with only a couple of levels. Each level is
usually an average over a region on the scalp, e.g.~left frontal
electrodes. This allows to test for specific contrasts in a step-down
fashion and is appropriate for setups that measure up to ca.~32
channels. For example, one first establishes overall significance for 
some global F-test and then tests differences or interactions. While
this procedure has its established place in the literature, it has the
disadvantage that one has to specify some contrast over
channels before testing. With high-density acquisitions, it is not
obvious which of the many possible averages will lead to the most
sensitive test. An alternative is to analyse the data in a
mass-univariate fashion over space and use GFT to assess
significance, adjusted for multiple comparisons, of topological
features like maxima. This approach has the advantage that one doesn't
have to specify expected locations of activations prior to the
test. This principle was recently demonstrated by Kilner et al. (2005)
for EEG power data.

Note that when using this mass-univariate approach, it is not possible
to test for region $\times$ condition interactions. By this we mean
that one tests for differences between voxels. For this one needs
multivariate models that take correlations over space directly into
account. In conventional ERP analyses testing for region $\times$
condition interactions is a standard procedure. So why can't you do
this in SPM? Although this issue sounds like a technical subtlety, the
main assumption behind such a test are vulnerable: When comparing
effects between voxels, one effectively assumes that the underlying
causes in different regions have the same effect on the scalp. This is
not necessarily true. 
