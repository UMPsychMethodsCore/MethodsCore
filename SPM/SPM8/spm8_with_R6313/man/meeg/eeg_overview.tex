\chapter{SPM for MEG/EEG overview \label{Chap:eeg:overview}}

\section{Welcome to SPM for M/EEG}

With SPM8 you can analyze all kinds of MEG and EEG data. Our group has recently published many articles about M/EEG analysis, in particular about Source Reconstruction\footnote{Source Reconstruction: \url{http://www.fil.ion.ucl.ac.uk/spm/doc/biblio/Keyword/EEG.html}} and Dynamic Causal Modelling\footnote{Dynamic Causal Modelling: \url{http://www.fil.ion.ucl.ac.uk/spm/doc/biblio/Keyword/DCM.html}} (DCM), which is a spatio-temporal network model to estimate effective connectivity in a network of sources. All these new methods, mostly based on a Bayesian approach, have been implemented in SPM8. We provide a range of tools for the full analysis pipeline, i.e., you can take your raw data from the MEG or EEG machine, and put it through SPM, starting from the conversion of the data through to a statistical analysis of sensor level or source reconstructed multi-subject data or Dynamic Causal Modelling.
\\
\\
Our overall goal is to provide an academic M/EEG analysis software package that can be used by everyone to apply the most recent methods available for the analysis of M/EEG data. As you may guess, this goal is quite ambitious because there is a large number of different M/EEG formats available, plus there are literally dozens of different analysis strategies that researchers would like to use. Clearly, our rather small group doesn't have the resources to cover all these different approaches. However, we made SPM for M/EEG as open as it possibly can be to allow researchers to use their favourite analysis software for specific processing steps. For example, it is possible to convert raw data to SPM8, then convert them to FieldTrip\footnote{FieldTrip: \url{http://www.ru.nl/neuroimaging/fieldtrip/}} or EEGLAB\footnote{EEGLAB: \url{http://sccn.ucsd.edu/eeglab/}} (using an SPM conversion routine), use a couple of functions in these packages, convert back to SPM, and do source reconstruction or DCM. Any combination of processing steps should be possible, and we expect that this software-interoperability among analysis software packages (each with its own area of expertise) will lead to a boost of M/EEG researchers trying out new ways of analysing their data with a wide range of sophisticated methods. We are pleased to say that we have a formal collaboration with the excellent FieldTrip package (head developer: Robert Oostenveld, F.C. Donders centre in Nijmegen/Netherlands) on many analysis issues. For example, SPM and FieldTrip share routines for converting data to \matlab, forward modelling for M/EEG source reconstruction and the SPM8 distribution contains a version of FieldTrip so that you can combine FieldTrip and SPM functions in your custom scripts. SPM and FieldTrip complement each other well, as SPM is geared toward very specific analysis tools as will be described below, whereas FieldTrip is a more general repository of different methods that can be put together in flexible ways to perform a variety of analyses. This flexibility of FieldTrip, however, comes at the expense of accessibility to a non-expert user. FieldTrip does not have a graphical user interface (GUI) and its functions are used by writing custom \matlab\ scripts. By combining SPM8 and FieldTrip the flexibility of FieldTrip can be complemented by SPM's GUI tools and batching system. Within this framework, power users can easily and rapidly develop specialized analysis tools with GUIs that can then  also be used by non-proficient \matlab\ users. Some examples of such tools are available in the MEEG toolbox distributed with SPM. We will also be happy to include in this toolbox new tools contributed by other users as long as they are of general interest and applicability.
\\
\\
SPM's speciality is, of course, the statistical analysis of voxel-based images. For statistical analysis, we use exactly the same routines as SPM for fMRI users would. These are robust and validated functions based on the General Linear Model\footnote{GLM: \url{http://www.fil.ion.ucl.ac.uk/spm/doc/biblio/Keyword/GLM.html}} (GLM) and Random Field Theory\footnote{RFT: \url{http://www.fil.ion.ucl.ac.uk/spm/doc/biblio/Keyword/RFT.html}} (RFT). These routines have been developed and used in the fMRI field over many years and are equally applicable to multi- (or single-) subject M/EEG studies.
\\
\\
Furthermore, our group has invested heavily in establishing Bayesian approaches to the source reconstruction of M/EEG data. Good source reconstruction techniques are vital for the M/EEG field, otherwise it would be very difficult to relate sensor data to neuroanatomy or findings from other modalities like fMRI. Bayesian source reconstruction provides a principled way of incorporating prior beliefs about how the data were generated, and enables principled methods for model comparison. With the use of priors and Bayesian model comparison, M/EEG source reconstruction is a very powerful neuroimaging tool, which has a unique macroscopic view on neuronal dynamics.

In addition, we have taken the idea of Dynamic Causal Modelling (DCM) from the fMRI domain, and applied it to the M/EEG field. For M/EEG, DCM is a powerful technique, because the data are highly resolved in time and this makes the identifiability of neurobiologically inspired network models feasible. This means that DCM can make inferences about temporal precedence of sources and can quantify changes in feedforward, backward and lateral connectivity among sources on a neuronal time-scale of milliseonds. Note that DCM/fMRI won't do this for you; DCM/fMRI (or any other connectivity analysis in fMRI) looks at the modulation of connectivity by task, on a time-scale of seconds.

\section{Changes from SPM5 to SPM8}

 As in SPM5, SPM8 provides tools for the analysis of EEG and MEG data. However, the SPM8 release is much more robust than the previous version, in many aspects such as conversion of data, source reconstruction, and dynamic causal modelling.
\\
\\
For three years, we have collected valuable experience for the analysis of M/EEG data, and received much valuable feedback from both FIL and external collaborators and users. We had plenty of opportunity to see which things worked well and what can be improved. One of our major insights was that writing a general routine for conversion of M/EEG data from their native to our SPM-format is a major effort. This is simply because there are so many different formats around and it is quite an undertaking for a small development team like ours to write stable software which can read all formats, some of which we have never seen ourselves. This had two consequences. The first is that we now collaborate with the developers of FieldTrip, who had already made available a wide range of \matlab\ code to convert M/EEG data, and we thought it a good idea to let both SPM and FieldTrip use and develop the same library. The second major difference is that we changed the internal M/EEG format of SPM in many ways to make reading/writing and manipulating M/EEG data more robust and straightforward for the user. Effectively, we invested a lot of effort into rebuilding the SPM for M/EEG machinery almost from scratch.
\\
\\
There are a couple of other major changes from SPM5 to SPM8.

First, based on our source reconstruction work, we have implemented a number of new routines which provide for a robust and efficient source reconstruction, using Bayesian approaches. The resulting, voxel-based source reconstructions can then be analysed, at the group level, with the same well-tested routines that are used for fMRI data.

Second, Dynamic Causal Modelling, a network analysis for spatiotemporal M/EEG data, has been developed further over the past three years. The DCM routines for modelling evoked responses or fields have been significantly improved both in functionality and speed. We now provide DCMs for modelling induced responses, phase coupling and local field potentials.

Third, there are now three ways of implementing M/EEG analyses in SPM. These are the graphical user interface, and two different scripting methods (batch analysis). These batch facilities come in handy for multi-subject studies. As in fMRI analysis, many processing steps are repetitive and it is now quite straightforward to automatize the software to a high degree.

Fourth, it is now possible to convert, in working memory, SPM data to FieldTrip or EEGLAB, and back. This feature makes it possible to use, within SPM, many FieldTrip and EEGLAB functions. For example, it is quite straightforward, using a script, to work within SPM, and use FieldTrip functions to do parts of the preprocessing.
\\
\\
The following chapters go through all the EEG/MEG related functionality of SPM8. Most users will probably find the tutorial (chapter \ref{Chap:data:mmn}) useful for a quick start. A more extensive tutorial demonstrating many new features of SPM8 on both EEG and MEG data can be foind in \ref{Chap:data:multimodal}. A further detailed description of the conversion, preprocessing functions, and the display is given in chapter \ref{Chap:eeg:preprocessing}. In chapter \ref{Chap:eeg:sensoranalysis}, we explain how one would use SPM's statistical machinery to analyse M/EEG data. The 3D-source reconstruction routines, including dipole modelling, are described in chapter \ref{Chap:eeg:imaging}. Finally, in chapter \ref{Chap:eeg:DCM}, we describe the graphical user interface for dynamical causal modelling, for evoked responses, induced responses, and local field potentials.
