\chapter{Developer's guide}
\label{Chap:batch:devel}

\section{SPM5 to Matlabbatch transition guide}

This is a short overview to describe code organisation and interfaces between
SPM and the batch system.

\subsection{Code Reorganisation}

The following paths have changed:
\begin{itemize}
\item \verb|fullfile(spm('dir'),'matlabbatch')| Core batch system.
\item \verb|fullfile(spm('dir'),'config')| New SPM config files.
\item \verb|fullfile(spm('dir'),'oldconfig')| Old SPM config files (unused)
\item \verb|spm_jobman.m| and \verb|spm_select.m| replaced with compatibility
  code
\item \verb|spm_Menu.fig| Callbacks adapted
\end{itemize}

Configuration code has been generated automatically from the existing SPM
configuration using \verb|cfg_struct2cfg| and \verb|gencode|. This sometimes
results in redundant/duplicate code. Also, conditional constructs like
\verb|if|, \verb|case| may not have been considered. 

Some assignments to configuration items are guarded by validity
checks. Usually, there will be a warning issued if a wrong value is
supplied. Special care needs to be taken for \verb|.prog|, \verb|.vfiles|,
\verb|.vout|, \verb|.check| functions or function handles. The functions
referenced here must be on MATLAB path before they are assigned to one of
these fields. For toolboxes, this implies that toolbox paths must be added at
the top of the configuration file.

\subsection{Interfaces between SPM and Matlabbatch}

\begin{description}
\item[Unchanged] harvested job structure.
\item[Changed] Top-level node in SPM config now called \verb|spmjobs| instead
  of \verb|jobs|. New overall top-level node
  \verb|matlabbatch|. \verb|spm_jobman| will convert and load SPM5 style batch
  jobs into the new batch system.
\item[Changed] Configuration file syntax - instead of structs, configuration
  items are now objects. Structs of type \verb|<type>| are now represented as
  objects of class \verb|cfg_<type>|. Existing SPM5 configuration can
  be imported using \verb|cfg_struct2cfg|. There is a new class
  \verb|cfg_exbranch| which is used for branches that have a \verb|.prog|
  field.
\item[Deprecated] Virtual files have been replaced by dependencies. These
  require computations to return a single output argument (e.g. a cell,
  struct). Parts of this output argument can be passed on to new inputs at
  run-time. Virtual files are treated as a special output argument.
\item[Added] Interface to the batch system
  \begin{itemize}
  \item \verb|cfg_util| Configuration management, job management, job
    execution
  \item \verb|cfg_serial| A utility to fill missing inputs and run a job
    (optionally with a GUI input function)
  \item \verb|cfg_ui| GUI - inspired by \verb|spm_jobman|, but modified to
    work around some MATLAB GUI ``features'' (like input widgets loosing focus
    before editing has finished).
  \end{itemize}

\end{description}

\section{Configuration Code Details}

Configuration code has been split into two files per configuration:
\begin{description}
\item[spm\_cfg\_*.m] Configuration classes, \verb|.check|, \verb|.vout|
  subfunctions
\item[spm\_run\_*.m] Run-time code, takes job structure as input and returns
  output structure as specified in \verb|.vout|.
\end{description}
In a few cases (where there was no additional magic in the code), run-time
code has been integrated into the main SPM code. This may be useful to run
test batches without using the configuration/batch system.

\subsection{Virtual Outputs}

Virtual outputs are described by arrays of \verb|cfg_dep| objects. These
objects contain a ``source'' and a ``target'' part. Functions may have more
than one virtual output (e.g. one output per session, a collection of results
variables). One \verb|cfg_dep| object has to be created for each output.

Only two fields in the ``source'' part need to be set in a \verb|.vout|
callback: 
\begin{description}
\item[sname] A display name for this output. This will appear in the
  dependencies list and should describe the contents of this dependency.
\item[src\_output] A subscript reference that can be used to address this
  output in the variable returned at run-time.
\item[tgt\_spec (optional)] A description on what kind of inputs this output
  should be displayed as dependency. This is not very convenient yet, the
  \verb|match| and \verb|cfg_findspec| methods are very restrictive in the
  kind of expressions that are allowed.
\end{description}

The \verb|.vout| callback will be evaluated once the configuration system
thinks that enough information about the \emph{structure} of the outputs is
available. This condition is met, once all in-tree nodes
\verb|cfg_(ex)branch|, \verb|cfg_choice|, \verb|cfg_repeat| have the required
number of child nodes. 

The \verb|.vout| callback is called with a job structure as input, but its
code \emph{should not rely} on the evaluation of any contents of this
structure (or at least provide a fallback). The contents of the leaf nodes may
not be set or may contain a dependency object instead of a value during
evalution of \verb|.vout|.

The ``target'' part will be filled by the configuration classes, the
\verb|src_exbranch| field is set in \verb|cfg_util|.

\subsection{SPM Startup}

The top level configuration file for SPM is \verb|spm_cfg.m|. It collects SPM
core configuration files and does toolbox autodetection. If a toolbox
directory contains \verb|*_cfg_*.m| files, they will be loaded. Otherwise, if
there are only SPM5-style \verb|*_config_*.m| files, the configuration will be
converted at run-time using \verb|cfg_struct2cfg|.

\subsection{Defaults Settings}

In Matlabbatch, there are different ways to set defaults:
\begin{enumerate}
\item \label{def:cf}in the configuration file itself,
\item \label{def:cd}in a defaults file, which has a structure similar to a harvested
  job,
\item \label{def:def}using a \verb|.def| field for leaf items.
\end{enumerate}
Defaults set using option \ref{def:cf} or \ref{def:cd} will only be
updated at SPM/matlabbatch startup. Defaults set using option
\ref{def:def} will be set once a new job is started. These defaults take
precedence over the other defaults.

In core SPM, these defaults refer to \verb|spm_get_defaults|, which accesses
\verb|spm_defaults|. Toolboxes may use their own callback functions.

\subsection{Toolbox Migration}

In the \verb|fullfile(spm('dir'),'toolbox')| folder there exists a migration
utility \verb|spm_tbx_config2cfg.m|. This utility will create a
\verb|*_cfg_*.m| and a \verb|*_def_*.m| file based on the configuration tree
given as input argument.

Toolboxes should set their defaults using the \verb|.def| fields,
using a mechanism similar to \verb|spm_get_defaults|. This allows for
flexibility without interfering with SPMs own defaults.

\section{Utilities}

\subsection{Batch Utilities}

Matlabbatch is designed to support multiple applications. A standard
application ``BasicIO'' is enabled by default. Among other options, it
contains file/file selection manipulation utilities which can be used as as
dependency source if multiple functions require the same set of files as input
argument. For debugging purposes, ``Pass Output to Workspace'' can be used to
assign outputs of a computation to a workspace variable.

The \verb|cfg_confgui| folder contains an application which describes all
configuration items in terms of configuration items. It is not enabled by
default, but can be added to the batch system using
\verb|cfg_util('addapp'...)|. This utility can be used generate a batch
configuration file with the batch system itself.

\subsection{MATLAB Code Generation}

The \verb|gencode| utility generates MATLAB \verb|.m| file code for any kind
of MATLAB variable. This is used to save batch files as well as to generate
configuration code.

\subsection{Configuration Management}

The backend utility to manage the configuration data is \verb|cfg_util|. It
provides callbacks to add application configurations, and to load, modify,
save or run jobs. These callbacks are used by two frontends: \verb|cfg_ui| is
a MATLAB GUI, while \verb|cfg_serial| can be used both as a GUI and in script
mode. In script mode, it will fill in job inputs from an argument list. This
allows to run predefined jobs with e.g. subject dependent inputs without
knowing the exact details of the job structure.
